\documentclass[b4paper,12pt,oneside,article]{memoir}
\setlrmarginsandblock{3cm}{*}{1}
\setulmarginsandblock{3cm}{*}{1}
\setheadfoot{2cm}{\footskip}          
\checkandfixthelayout[nearest]


\setlength{\parindent}{0pt}

\usepackage[utf8]{inputenc} % 显示中文
\usepackage{fancyhdr} % 导入fancyhdr包
\usepackage{amsmath, amssymb, bm, mathtools, mathdots}

\pagestyle{fancy}
% 页眉设置
\fancyhf{}
\fancyhead[L]{Yilin Chen}
\fancyhead[R]{\today}
\fancyhead[C]{\rightmark}
% 页脚设置F8
\fancyfoot[L]{Powered by \LaTeX}
\fancyfoot[C]{\thepage} % 页码
\fancyfoot[R]{DM Homework Assignment}
\renewcommand{\headrulewidth}{4pt} % 分隔线宽度4磅
\renewcommand{\footrulewidth}{2pt}

\usepackage{newtxtext}
\usepackage{geometry}
\usepackage[dvipsnames,svgnames]{xcolor}
% \usepackage[strict]{changepage} % 提供一个 adjustwidth 环境
%\usepackage{framed} % 实现方框效果
\usepackage{tcolorbox}
\usepackage{float}
\usepackage{shortlst}
\usepackage{makecell}
\usepackage{graphicx}
\usepackage{caption}
\setcounter{secnumdepth}{0}

 %-------------------------
 \usepackage{tabularx}

 \newcounter{proofstep}
 \newcommand{\step}{}
 \newenvironment{proof}
   {\par
	\setcounter{proofstep}{0}%
	\renewcommand{\step}{\refstepcounter{proofstep}\theproofstep. &}%
	\noindent
	\begin{tabular}{ @{} r c l @{}}}
   {\end{tabular}}
 %-------------------------

\definecolor{formalshade}{rgb}{0.95,0.95,1} % 文本框颜色
\definecolor{greenshade}{rgb}{0.90,0.99,0.91} % 绿色文本框,竖线颜色设为 Green
\definecolor{redshade}{rgb}{1.00,0.90,0.90}% 红色文本框,竖线颜色设为 LightCoral
\definecolor{brownshade}{rgb}{0.99,0.97,0.93} % 莫兰迪棕色,竖线颜色设为 BurlyWood

\newenvironment{formal}{%
\def\FrameCommand{%
\hspace{1pt}%
{\color{DarkBlue}\vrule width 2pt}%
{\color{formalshade}\vrule width 4pt}%
\colorbox{formalshade}%
}%

\MakeFramed{\advance\hsize-\width\FrameRestore}%
\noindent\hspace{-4.55pt}
\begin{adjustwidth}{}{7pt}%
\vspace{2pt}\vspace{2pt}%
}
{%
\vspace{2pt}\end{adjustwidth}\endMakeFramed%
}

\usepackage[framemethod=TikZ]{mdframed}

\newcounter{Sol}[section]
\renewcommand{\theSol}{\arabic{section}.\arabic{Sol}}
\newenvironment{Sol}[1][]{
	\refstepcounter{Sol}
	\mdfsetup{
		frametitle={
			\tikz[baseline=(current bounding box.east), outer sep=0pt]
			\node[anchor=east,rectangle,fill=blue!20]
			{\strut Solution~\ifstrempty{#1}{}{:~#1}};},
		innertopmargin=10pt,linecolor=blue!20,
		linewidth=2pt,topline=true,
		frametitleaboveskip=\dimexpr-\ht\strutbox\relax
	}
	\begin{mdframed}[]\relax
}{\end{mdframed}}


\begin{document}
\chapter*{Chapter 2 Basic Structures: Sets, Functions,
Sequences, Sums, and Matrices}
\section{2.1 Sets}

%---------------------------------------------------------------------------

\begin{tcolorbox}
	[colback=Emerald!10,colframe=cyan!40!black,title=\textbf{Question 6}]
	For each of these pairs of sets, determine whether the first
    is a subset of the second, the second is a subset of the first,
    or neither is a subset of the other.
    \begin{enumerate}[a)]
        \item the set of people who speak English, the set of people who speak English with an Australian accent
        \item the set of fruits, the set of citrus fruits
        \item the set of students studying discrete mathematics, the set of students studying data structures
    \end{enumerate}
    
\end{tcolorbox}
\begin{Sol}[]
	\begin{enumerate}[a)]
        \item $a\nsubseteq b$
        \item $a\nsubseteq b$
        \item $a\nsubseteq b$
    \end{enumerate}
\end{Sol}

%---------------------------------------------------------------------------

\clearpage

%---------------------------------------------------------------------------

\begin{tcolorbox}
	[colback=Emerald!10,colframe=cyan!40!black,title=\textbf{Question 12}]
	Determine whether these statements are true or False.
    \begin{enumerate}[a)]
        \item  $\varnothing  \in  \{\varnothing \} $
        \item $\varnothing  \in \{\varnothing , \{\varnothing \}\}$
        \item $\{\varnothing \} \in \{\varnothing \} $
        \item $\{\varnothing \} \in \{\{\varnothing \}\}$
        \item $\{\varnothing \} \subset  \{\varnothing , \{\varnothing \}\} $
        \item $\{\{\varnothing \}\} \subset \{\varnothing , \{\varnothing \}\}$
        \item $\{\{\varnothing \}\} \subset \{\{\varnothing \}, \{\varnothing \}\}$
    \end{enumerate}
    
    
\end{tcolorbox}
\begin{Sol}[]
	\begin{enumerate}[a)]
        \item True
        \item True
        \item False
        \item True
        \item True
        \item True
        \item False
    \end{enumerate}
\end{Sol}

%---------------------------------------------------------------------------

%---------------------------------------------------------------------------

\begin{tcolorbox}
	[colback=Emerald!10,colframe=cyan!40!black,title=\textbf{Question 20}]
	Find two sets A and B such that $A \in B$ and $A \subseteq B$.
\end{tcolorbox}
\begin{Sol}[]
	$A=\emptyset$ and $B=\{ \emptyset \}$
\end{Sol}

%---------------------------------------------------------------------------

\clearpage

%---------------------------------------------------------------------------

\begin{tcolorbox}
	[colback=Emerald!10,colframe=cyan!40!black,title=\textbf{Question 22}]
	What is the cardinality of each of these sets?
    \begin{enumerate}[a)]
        \item $\emptyset $
        \item $\{\emptyset\}$
        \item $\{\emptyset, \{\emptyset\}\} $
        \item $\{\emptyset, \{\emptyset\}, \{\emptyset, \{\emptyset\}\}\}$
    \end{enumerate}
\end{tcolorbox}
\begin{Sol}[]
	\begin{enumerate}[a)]
        \item 0
        \item 1
        \item 2
        \item 3
    \end{enumerate}
\end{Sol}

%---------------------------------------------------------------------------

\clearpage

%---------------------------------------------------------------------------

\begin{tcolorbox}
	[colback=Emerald!10,colframe=cyan!40!black,title=\textbf{Question 26}]
	Determine whether each of these sets is the power set of
    a set, where a and b are distinct elements.
    \begin{enumerate}[a)]
        \item $\emptyset $
        \item $\{\emptyset, \{a\} \} $
        \item $\{\emptyset, \{a\} , \{\emptyset, a\} \}  $
        \item $\{\emptyset, \{a\} , \{b\} , \{a, b\} \} $
    \end{enumerate}
\end{tcolorbox}
\begin{Sol}[]
	\begin{enumerate}[a)]
        \item Yes
        \item Yes
        \item No
        \item Yes
    \end{enumerate}
\end{Sol}

%---------------------------------------------------------------------------

\clearpage

%---------------------------------------------------------------------------

\begin{tcolorbox}
	[colback=Emerald!10,colframe=cyan!40!black,title=\textbf{Question 28}]
	Show that if $A \subseteq C$ and $B \subseteq D$, then $A \times  B \subseteq C \times  D$
\end{tcolorbox}
\begin{Sol}[]
	For $A \times  B$, supposse a pair $(a,b)$ where $a\in A$ and $b \in B$, since A is a subset of C and B is a subset of D, so $a \in C$ and $b \in D$, Therefore $(a,b) \in C \times D$, and $A \times  B \subseteq C \times  D$.
\end{Sol}

%---------------------------------------------------------------------------

%---------------------------------------------------------------------------

\begin{tcolorbox}
	[colback=Emerald!10,colframe=cyan!40!black,title=\textbf{Question 32}]
	Suppose that $A \times  B = \emptyset$, where A and B are sets. What
    can you conclude?
\end{tcolorbox}
\begin{Sol}[]
	$A = \emptyset$ or $b = \emptyset$
\end{Sol}

%---------------------------------------------------------------------------

%---------------------------------------------------------------------------

\begin{tcolorbox}
	[colback=Emerald!10,colframe=cyan!40!black,title=\textbf{Question 38}]
	How many different elements does $A \times B \times C$ have if A
    has m elements, B has n elements, and C has p elements?
\end{tcolorbox}
\begin{Sol}[]
	$A \times B \times C$ has $mnp$ elements.
\end{Sol}

%---------------------------------------------------------------------------
\clearpage

\section{2.2 Set Operations}

%---------------------------------------------------------------------------

\begin{tcolorbox}
	[colback=Emerald!10,colframe=cyan!40!black,title=\textbf{Question 4}]
	Let $A = \{ a, b, c, d, e\}$ and$ B = \{ a, b, c, d, e, f, g, h\}$. Find
    \begin{enumerate}[a)]
        \item $A \cup B. $
        \item $A \cap B.$
        \item $A - B. $
        \item $B - A.$
    \end{enumerate}
\end{tcolorbox}
\begin{Sol}[]
	\begin{enumerate}[a)]
        \item $ \{ a, b, c, d, e, f, g, h\}$
        \item $ \{ a, b, c, d, e\}$
        \item $ \emptyset$
        \item$ \{ f,g,h\}$
    \end{enumerate}
\end{Sol}

%---------------------------------------------------------------------------
%---------------------------------------------------------------------------

\begin{tcolorbox}
	[colback=Emerald!10,colframe=cyan!40!black,title=\textbf{Question 12}]
	Prove the first absorption law from Table 1 by showing
    that if A and B are sets, then $A \cup (A \cap B) = A.$
\end{tcolorbox}
\begin{Sol}[]
    Suppose $x \in A $, then $x \in A \cup (A \cap B)$ since $A \cup (A \cap B)$  is a union of $A$ and another set, so $A \cup (A \cap B) \subseteq A$. \  
    Suppose $x \in A \cup (A \cap B)$, then $x \in A$ or $x \in (A \cap B)$ by the defintion of union,  if $x \in (A \cap B)$ then $x \in A$ by the defintion of intersection. Therefore, $A \cup (A \cap B) = A.$
\end{Sol}

%---------------------------------------------------------------------------

\clearpage

%---------------------------------------------------------------------------

\begin{tcolorbox}
	[colback=Emerald!10,colframe=cyan!40!black,title=\textbf{Question 16}]
	Let A and B be sets. Show that
    \begin{enumerate}[a)]
        \item $(A \cap B) \subseteq A. $
        \item $A \subseteq (A \cup B).$
        \item $A - B \subseteq A. $
        \item $A \cap (B - A) = \emptyset.$
        \item $A \cup (B - A) = A \cup B.$
    \end{enumerate}
\end{tcolorbox}
\begin{Sol}[]
    \begin{enumerate}[a)]
        \item If x is in $A \cap B$, then perforce it is in A (by definition of intersection).
        \item If x is in A, then perforce it is in$ A \cup B$ (by definition of union).
        \item If x is in $A - B$, then perforce it is in A (by definition of difference).
        \item If $x \in A$ then $x \in B - A$. Therefore there can be no elements in $A \cap (B - A)$, so $A \cap (B - A) = \emptyset$.
        \item $A \cup (B - A) = A \cup B.$ means $x \in A$ or $x \in B \ AND\ x\notin A$, which means $x \in A \ OR\ x \in B$. Therefore $A \cup (B - A) = A \cup B.$
    \end{enumerate}
\end{Sol}

%---------------------------------------------------------------------------


%---------------------------------------------------------------------------

\begin{tcolorbox}
	[colback=Emerald!10,colframe=cyan!40!black,title=\textbf{Question 26}]
	Let A, B, and C be sets. Show that $(A - B) - C =(A - C) - (B - C).$
\end{tcolorbox}
\begin{Sol}[]
    Suppose x is in the left side, then x must be in A but in neither B nor C . Thus $x \in A - C $,
    but $x \notin B - C$ , so x is in the right-hand side. \\
    Suppose x is in the right side, then x must be in $A-C$ but not in $B-C$ . Thus $x \in A-C $,
    but $x \notin B$ , so x is in the left-hand side. 
\end{Sol}

%---------------------------------------------------------------------------

\clearpage

%---------------------------------------------------------------------------

\begin{tcolorbox}
	[colback=Emerald!10,colframe=cyan!40!black,title=\textbf{Question 32}]
	Can you conclude that A = B if A, B, and C are sets such that
    \begin{enumerate}[a)]
        \item $A \cup C = B \cup C? $
        \item $A \cap C = B \cap C?$
        \item $A \cup C = B \cup C and A \cap C = B \cap C?$
    \end{enumerate}
    
\end{tcolorbox}
\begin{Sol}[]
    \begin{enumerate}[a)]
        \item No
        \item No
        \item Yes,By symmetry, it suffices to prove that $A \subseteq B$. Suppose that $x \in A$. There are two cases. If $x \in C$ , then $x \in A \cap C = B \cap C$ , whichforces $ \in B$. On the other hand, if $x \notin C $, then because $x \in A \cup C = B \cup C$ , we must have $x \in B$.        
    \end{enumerate}
\end{Sol}

%---------------------------------------------------------------------------

\clearpage

%---------------------------------------------------------------------------

\begin{tcolorbox}
	[colback=Emerald!10,colframe=cyan!40!black,title=\textbf{Question 56}]
	Find $\bigcup^{\infty }_{i=1} A_i$ and $\bigcap^{\infty }_{i=1} A_i$if for every positive integer i,

    \begin{enumerate}[a)]
        \item $A_i = {i, i + 1, i + 2, …}.$
        \item $A_i = {0, i}.$
        \item $A_i = (0, i)$, that is, the set of real numbers x with $0 < x < i.$
        \item $A_i = (i, ∞)$, that is, the set of real numbers x with $x > i.$
    \end{enumerate}
    
\end{tcolorbox}
\begin{Sol}[]
    \begin{enumerate}[a)]
        \item $\bigcup^{\infty }_{i=1} A_i=\{1,2,3,4,...\}$ and $\bigcap^{\infty }_{i=1} A_i=\emptyset$ since Every positive integer is excluded from at least one of the sets
        \item $\bigcup^{\infty }_{i=1} A_i=\{0,1,2,3,4,...\}$ and $\bigcap^{\infty }_{i=1} A_i=\{0\}$ 
        \item $\bigcup^{\infty }_{i=1} A_i=\mathbb{R} ^+$ and $\bigcap^{\infty }_{i=1} A_i=A_1$ 
        \item $\bigcup^{\infty }_{i=1} A_i=A_1$ and $\bigcap^{\infty }_{i=1} A_i=\emptyset$ 
    \end{enumerate}
\end{Sol}

%---------------------------------------------------------------------------

\clearpage

%---------------------------------------------------------------------------

\begin{tcolorbox}
	[colback=Emerald!10,colframe=cyan!40!black,title=\textbf{Question 58}]
	Suppose that the universal set is $U = \{1, 2, 3, 4,5, 6, 7, 8, 9, 10\}$. Express each of these sets with bitstrings where the ith bit in the string is 1 if i is in theset and 0 otherwise.


    \begin{enumerate}[a)]
        \item $\{3, 4, 5\}$
        \item $\{1, 3, 6, 10\}$
        \item $\{2, 3, 4, 7, 8, 9\}$
    \end{enumerate}
    
\end{tcolorbox}
\begin{Sol}[]
    \begin{enumerate}[a)]
        \item 0011100000
        \item 1010010001
        \item 0111001110
    \end{enumerate}
\end{Sol}

%---------------------------------------------------------------------------

\clearpage

\section{2.1 Functions}

%---------------------------------------------------------------------------

\begin{tcolorbox}
	[colback=Emerald!10,colframe=cyan!40!black,title=\textbf{Question 2}]
	Determine whether \textit{f} is a function from $\mathbb{Z} $ to $\mathbb{R} $ if
    \begin{enumerate}[a)]
        \item $f(n) =\pm n.$
        \item $f(n) =\sqrt{n^2 + 1}.$
        \item $f(n) = 1/(n^2 - 4).$
    \end{enumerate}
    
\end{tcolorbox}
\begin{Sol}[]
    \begin{enumerate}[a)]
        \item No. Because $f(n)$ is not an assignment of exactly one element of $\mathbb{R} $ to each element of $\mathbb{Z} $.
        \item Yes.
        \item No. Because $f(2)$ and $f(-2)$ are not defined.
    \end{enumerate}
\end{Sol}

%---------------------------------------------------------------------------

\clearpage

%---------------------------------------------------------------------------

\begin{tcolorbox}
	[colback=Emerald!10,colframe=cyan!40!black,title=\textbf{Question 4}]
	Find the domain and range of these functions. Note that
    in each case, to find the domain, determine the set of elements assigned values by the function.
    \begin{enumerate}[a)]
        \item the function that assigns to each nonnegative integer its last digit
        \item the function that assigns the next largest integer to a positive integer
        \item the function that assigns to a bit string the number of one bits in the string
        \item the function that assigns to a bit string the number of bits in the string
    \end{enumerate}
\end{tcolorbox}
\begin{Sol}[]
    \begin{enumerate}[a)]
        \item The domain is the set of nonnegative integers, and the range is the set of digits ( 0 through 9 ).
        \item The domain is the set of positive integers, and the range is the set of positive integers greater than 1.
        \item The domain is the set of strings, and the range is the set of nonnegative integers.
        \item The domain is the set of strings, and the range is the set of nonnegative integers.
    \end{enumerate}
\end{Sol}

%---------------------------------------------------------------------------

\clearpage

%---------------------------------------------------------------------------

\begin{tcolorbox}
	[colback=Emerald!10,colframe=cyan!40!black,title=\textbf{Question 8}]
	Find these values.

    \begin{enumerate}[a)]
        \item  $\lfloor 1.1\rfloor $
        \item $\lceil  1.1\rceil  $
        \item $\lfloor -0.1\rfloor $
        \item $\lceil  -0.1\rceil  $
        \item $\lceil  2.99\rceil  $
        \item $\lceil  -2.99\rceil  $
        \item $\lfloor \frac{1}{2}+\lceil  \frac{1}{2}\rceil \rfloor $
        \item $\lceil \lfloor  \frac{1}{2}\rfloor+\lceil  \frac{1}{2}\rceil + \frac{1}{2}\rceil $
    \end{enumerate}
\end{tcolorbox}
\begin{Sol}[]
    \begin{enumerate}[a)]
        \item 1
        \item 2
        \item -1
        \item 0
        \item 3
        \item -2
        \item 1
        \item 2
    \end{enumerate}
\end{Sol}

%---------------------------------------------------------------------------\

\clearpage

%---------------------------------------------------------------------------

\begin{tcolorbox}
	[colback=Emerald!10,colframe=cyan!40!black,title=\textbf{Question 16}]
	Consider these functions from the set of students in a discrete mathematics class. Under what conditions is the function one-to-one if it assigns to a student his or her
    \begin{enumerate}[a)]
        \item mobile phone number.
        \item student identification number.
        \item final grade in the class.
        \item home town.
    \end{enumerate}
\end{tcolorbox}
\begin{Sol}[]
    \begin{enumerate}[a)]
        \item This is normally the one-to-one function.
        \item This is  one-to-one function.
        \item This is normally not the one-to-one function since it is likely that two students have final grade in the class.
        \item This is normally not the one-to-one function since it is likely that two students came from same home towm.
    \end{enumerate}
\end{Sol}

%---------------------------------------------------------------------------

\clearpage

%---------------------------------------------------------------------------

\begin{tcolorbox}
	[colback=Emerald!10,colframe=cyan!40!black,title=\textbf{Question 30}]
	Let $S = \{ -1, 0, 2, 4, 7\} $. Find $f(S)$ if
    
    \begin{enumerate}[a)]
        \item $f(x) = 1. $
        \item $f(x) = 2x + 1.$
        \item $f(x) = \lceil x/5\rceil. $
        \item $f(x)=\lfloor(x^2 + 1)∕3\rfloor.$
    \end{enumerate}
\end{tcolorbox}
\begin{Sol}[]
    \begin{enumerate}[a)]
        \item $f(S) = \{ 1\}. $
        \item $f(x) = \{ -1,1,5,9,15\}.$
        \item $f(x) = \{ 0,1,2\}. $
        \item $f(x)=\{ 0,1,5,16\}.$
    \end{enumerate}
\end{Sol}

%---------------------------------------------------------------------------

\clearpage

%---------------------------------------------------------------------------

\begin{tcolorbox}
	[colback=Emerald!10,colframe=cyan!40!black,title=\textbf{Question 44}]
	Let f be the function from $\mathbb{R}$ to $\mathbb{R}$ defined by $f(x) = x^2$. Find
    
    \begin{enumerate}[a)]
        \item $f^{ -1}({1}). $
        \item $f^{ -1}({x | 0 < x < 1}).$
        \item $f^{ -1}({x | x > 4}).$
    \end{enumerate}
\end{tcolorbox}
\begin{Sol}[]
    \begin{enumerate}[a)]
        \item $f^{ -1}({1}) = \{ 1,-1\}. $
        \item $f^{ -1}({x | 0 < x < 1})=\{ x | 0 < x < 1 \ \lor \ -1<x<0\}.$
        \item $f^{ -1}({x | x > 4})=\{ x | x >2 \ \lor \ x < -2 \}.$
    \end{enumerate}
\end{Sol}

%---------------------------------------------------------------------------
%---------------------------------------------------------------------------

\begin{tcolorbox}
	[colback=Emerald!10,colframe=cyan!40!black,title=\textbf{Question 66}]
	Draw the graph of the function $f(x) = \lfloor x/2 \rfloor$ from $\mathbb{R}$ to $\mathbb{R}$.
\end{tcolorbox}
\begin{Sol}[]
    \includegraphics*[scale=0.6]{fun.png}
\end{Sol}

%---------------------------------------------------------------------------

\clearpage

%---------------------------------------------------------------------------

\begin{tcolorbox}
	[colback=Emerald!10,colframe=cyan!40!black,title=\textbf{Question 78}]
	Let x be a real number. Show that $\lfloor 3x\rfloor  =\lfloor x\rfloor + \lfloor x + \frac{1}{3} \rfloor  + \lfloor x + \frac{2}{3}\rfloor $.
\end{tcolorbox}
\begin{Sol}[]
    Suppose $p \leq x<p+1$, then $\lfloor x\rfloor=p$.
    \begin{itemize}
        \item If $p\leq x < p+\frac{1}{3}$, then $\lfloor 3x\rfloor=3p$, $\lfloor x\rfloor + \lfloor x + \frac{1}{3} \rfloor  + \lfloor x + \frac{2}{3}\rfloor =p+p+p=3p$
        \item If $p+\frac{1}{3}\leq x < p+\frac{2}{3}$, then $\lfloor 3x\rfloor=3p+1$, $\lfloor x\rfloor + \lfloor x + \frac{1}{3} \rfloor  + \lfloor x + \frac{2}{3}\rfloor =p+p+(p+1)=3p+1$
        \item If $p+\frac{2}{3}\leq x < p+1$, then $\lfloor 3x\rfloor=3p+2$, $\lfloor x\rfloor + \lfloor x + \frac{1}{3} \rfloor  + \lfloor x + \frac{2}{3}\rfloor =p+(p+1)+(p+1)=3p+2$
    \end{itemize}
    Therefore, $\lfloor 3x\rfloor  =\lfloor x\rfloor + \lfloor x + \frac{1}{3} \rfloor  + \lfloor x + \frac{2}{3}\rfloor $.
\end{Sol}

%---------------------------------------------------------------------------

\clearpage

\section{2.4 Sequences and Summations}

%---------------------------------------------------------------------------

\begin{tcolorbox}
	[colback=Emerald!10,colframe=cyan!40!black,title=\textbf{Question 4}]
	What are the terms $a_0, a_1, a_2$, and $a_3$ of the sequence $\{ a_n \}$,where an equals
    \begin{enumerate}[a)]
        \item $(-2)^n$? 
        \item $3$?
        \item $7 + 4^n$? 
        \item $2^n + (-2)^n$?
    \end{enumerate}

\end{tcolorbox}
\begin{Sol}[]
    \begin{enumerate}[a)]
        \item $a_0=1, a_1=-2, a_2=4,a_3=-8$
        \item $a_0=3, a_1=3, a_2=3,a_3=3$
        \item $a_0=8, a_1=11, a_2=23,a_3=71$
        \item $a_0=2, a_1=0, a_2=8,a_3=0$
    \end{enumerate}
\end{Sol}

%---------------------------------------------------------------------------

\clearpage

%---------------------------------------------------------------------------

\begin{tcolorbox}
	[colback=Emerald!10,colframe=cyan!40!black,title=\textbf{Question 10}]
	Find the first six terms of the sequence defined by each of these recurrence relations and initial conditions.

    \begin{enumerate}[a)]
        \item $a_n = -2a_{n-1}, a_0 = -1$
        \item $a_n = a_{n-1} - a_{n-2}, a_0 = 2, a_1 = -1$
        \item $a_n = 3a^2_{n-1}, a_0 = 1$
        \item $a_n = na_{n-1} + a^2_{n-2}, a_0 = -1, a_1 = 0$
        \item $a_n = a_{n-1} - a_{n-2} + a_{n-3}, a_0 = 1, a_1 = 1, a_2 = 2$
    \end{enumerate}

\end{tcolorbox}
\begin{Sol}[]
    \begin{enumerate}[a)]
        \item $a_0=-1, a_1=2, a_2=-4,a_3=8,a_4=-16,a_5=32$
        \item $a_0=2, a_1=-1, a_2=-3,a_3=-2,a_4=1,a_5=3$
        \item $a_0=1, a_1=3, a_2=27,a_3=2187,a_4=14348907,a_5=617673396283947$
        \item $a_0=1, a_1=1, a_2=2,a_3=2,a_4=1,a_5=1$
    \end{enumerate}
\end{Sol}

%---------------------------------------------------------------------------

\clearpage

%---------------------------------------------------------------------------

\begin{tcolorbox}
	[colback=Emerald!10,colframe=cyan!40!black,title=\textbf{Question 20}]
	Assume that the population of the world in 2017 was 7.6 billion and is growing at the rate of 1.12\% a year.
    \begin{enumerate}[a)]
        \item Set up a recurrence relation for the population of the world $n$ years after 2017.
        \item Find an explicit formula for the population of the world n years after 2017.
        \item What will the population of the world be in 2050?
    \end{enumerate}

\end{tcolorbox}
\begin{Sol}[]
    Suppose the population of the world n years after 2017 is $a_n$ (billions).
    \begin{enumerate}[a)]
        \item $a_n=(100+1.12)\% \cdot a_{n-1}, a_0=7.6$
        \item $a_n=7.6\cdot[(100+1.12)\%]^n$
        \item $a_{33} \approx 10.98 $
    \end{enumerate}
\end{Sol}

%---------------------------------------------------------------------------

\clearpage

%---------------------------------------------------------------------------

\begin{tcolorbox}
	[colback=Emerald!10,colframe=cyan!40!black,title=\textbf{Question 24}]
	
    \begin{enumerate}[a)]
        \item Find a recurrence relation for the balance $B(k)$ owed at the end of k months on a loan at a rate of $r$ if a payment $P$ is made on the loan each month. [\textit{Hint}: Express $B(k)$ in terms of $B(k -1)$ and note that the monthly interest rate is $r/12$.]
        \item Determine what the monthly payment $P$ should be so that the loan is paid off after $T$ months.
    \end{enumerate}

\end{tcolorbox}
\begin{Sol}[]
    \begin{enumerate}[a)]
        \item $B(k)=(1+r/12)B(k-1)-P$
        \item From recurrence relation we can know the explicit formula is $B(k)-12P/r=(1+r/12)^k(B(0)-12P/r)$\\
        Then we can solve that $T=\log_{1+r/12}(\frac{-12p}{B(0)r-12p})$
    \end{enumerate}
\end{Sol}

%---------------------------------------------------------------------------

\clearpage

%---------------------------------------------------------------------------

\begin{tcolorbox}
	[colback=Emerald!10,colframe=cyan!40!black,title=\textbf{Question 32}]
	Find the value of each of these sums.

    \begin{enumerate}[a)]
        \item $\Sigma_{j=0}^{8}=(-1+(-1)^j)$
        \item $\Sigma_{j=0}^{8}=(3^j-2^j)$
        \item $\Sigma_{j=0}^{8}=(2 \cdot 3^j+3 \cdot2^j)$
        \item $\Sigma_{j=0}^{8}=(2^{j+1}-2{^j})$
    \end{enumerate}

\end{tcolorbox}
\begin{Sol}[]
    \begin{enumerate}[a)]
        \item $\Sigma_{j=0}^{8}=(-1+(-1)^j)=0$
        \item $\Sigma_{j=0}^{8}=(3^j-2^j)=9330$
        \item $\Sigma_{j=0}^{8}=(2 \cdot 3^j+3 \cdot2^j)=21215$
        \item $\Sigma_{j=0}^{8}=(2^{j+1}-2{^j})=511$
    \end{enumerate}
\end{Sol}

%---------------------------------------------------------------------------
\end{document}